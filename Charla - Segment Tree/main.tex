\documentclass{beamer}

\usepackage[utf8]{inputenc}     %encoding del codigo fuente
\usepackage[T1]{fontenc}        %encoding del pdf
\usepackage[spanish]{babel}     %lenguaje para cortes de palabras, bibliografia, etc
\usepackage{graphicx}
\usepackage{listings}           %Codigo: chequear minted
%\usepackage{tcolorbox}
%\tcbuselibrary{minted,breakable,skins}
\usepackage{tikz}

\usetheme{Boadilla}
\setlength{\parskip}{\baselineskip}

\title{Range Queries Con Updates}
\subtitle{Segment Tree y otras variantes}
\author{Ulises López Pacholczak}
\date[PC2]{Programación Competitiva 2\\\today}

\definecolor{keywords}{HTML}{44548A}
\definecolor{strings}{HTML}{00999A}
\definecolor{comments}{HTML}{990000}

\lstset{language=C++,frame=single,basicstyle=\ttfamily \small,showstringspaces=false,columns=flexible}
\lstset{
  literate={ö}{{\"o}}1
           {ä}{{\"a}}1
           {ü}{{\"u}}1
}
\lstset{xleftmargin=0pt,xrightmargin=0pt}
\lstset{aboveskip=12pt,belowskip=8pt}

\lstset{
    commentstyle=\color{comments},
    keywordstyle=\color{keywords},
    stringstyle=\color{strings}
}

\AtBeginSection[]
{
  \begin{frame}
    \tableofcontents[currentsection,hideallsubsections]
  \end{frame}
}


\begin{document}

\begin{frame}
\titlepage
\end{frame}

\begin{frame}
\small
\tableofcontents[hideallsubsections]
\end{frame}

\section{Introducción}

\begin{frame}{Introducción}
Ya sabemos como responder queries de suma en rango, en $O(n)$ tiempo de procesamiento ($O(n*m)$ si es una tabla) y $O(1)$ tiempo de consulta.

\pause
También, se pueden resolver queries de máximo o mínimo en rangos con $O(n \log n)$ tiempo de procesamiento y $O(1)$ tiempo de consulta.

\pause
¿Pero esto es suficiente para todos los problemas?
\end{frame}

\section{Segment Tree}

\begin{frame}{Suma en Rangos con Updates}

\begin{block}{CSES 1648 - Dynamic Range Sum Queries}
Dado un arreglo de $n$ enteros, tu tarea es procesar $q$ queries de los siguientes tipos:
\begin{enumerate}
    \item Actualizar el valor en la posición $k$ a un valor $u$
    \item Cual es la suma de los valores en el rango $[a,b]$?
\end{enumerate}
Tanto $q$ y $n$ pueden ser como máximo $2*10^5$
\end{block}

\url{https://cses.fi/problemset/task/1648}
\end{frame}

\begin{frame}{Con lo que ya sabemos...}
Por cada query de tipo $1$, debemos realizar un reprocesamiento de $O(n)$. Por ende, la complejidad del problema nos quedaría $O(n^2)$, lo mismo que hacerlo con fuerza bruta.

\pause
Es claro que necesitamos otra estructura de datos para poder manejar los updates de forma eficiente...
\end{frame}

\subsection{Explicación de la Estructura}
\begin{frame}{Segment Tree}
El Segment Tree es una estructura de datos que nos permite realizar queries en rangos y updates en $O(\log n)$.
\pause

\begin{center}
\begin{tikzpicture}[scale=0.7]
\draw (0,0) grid (8,1);

\node[anchor=center] at (0.5, 0.5) {5};
\node[anchor=center] at (1.5, 0.5) {8};
\node[anchor=center] at (2.5, 0.5) {6};
\node[anchor=center] at (3.5, 0.5) {3};
\node[anchor=center] at (4.5, 0.5) {2};
\node[anchor=center] at (5.5, 0.5) {7};
\node[anchor=center] at (6.5, 0.5) {2};
\node[anchor=center] at (7.5, 0.5) {6};

\node[draw, circle] (a) at (1,2.5) {13};
\path[draw,thick,-] (a) -- (0.5,1);
\path[draw,thick,-] (a) -- (1.5,1);
\node[draw, circle,minimum size=22pt] (b) at (3,2.5) {9};
\path[draw,thick,-] (b) -- (2.5,1);
\path[draw,thick,-] (b) -- (3.5,1);
\node[draw, circle,minimum size=22pt] (c) at (5,2.5) {9};
\path[draw,thick,-] (c) -- (4.5,1);
\path[draw,thick,-] (c) -- (5.5,1);
\node[draw, circle,minimum size=22pt] (d) at (7,2.5) {8};
\path[draw,thick,-] (d) -- (6.5,1);
\path[draw,thick,-] (d) -- (7.5,1);

\node[draw, circle] (i) at (2,4.5) {22};
\path[draw,thick,-] (i) -- (a);
\path[draw,thick,-] (i) -- (b);
\node[draw, circle] (j) at (6,4.5) {17};
\path[draw,thick,-] (j) -- (c);
\path[draw,thick,-] (j) -- (d);

\node[draw, circle] (m) at (4,6.5) {39};
\path[draw,thick,-] (m) -- (i);
\path[draw,thick,-] (m) -- (j);
\end{tikzpicture}
\end{center}

{\tiny En este caso, cada nodo guarda la sumatoria de sus descendientes}

\end{frame}

\begin{frame}{Funcionamiento}
Cada nodo representa el valor de realizar una operación (en el caso del problema original, la suma) en un rango del arreglo.

\begin{center}
\begin{tikzpicture}[scale=0.7]
%\fill[color=lightgray] (1,0) rectangle (7,1);
\draw (0,0) grid (8,1);

\node[anchor=center] at (0.5, 0.5) {[0]};
\node[anchor=center] at (1.5, 0.5) {[1]};
\node[anchor=center] at (2.5, 0.5) {[2]};
\node[anchor=center] at (3.5, 0.5) {[3]};
\node[anchor=center] at (4.5, 0.5) {[4]};
\node[anchor=center] at (5.5, 0.5) {[5]};
\node[anchor=center] at (6.5, 0.5) {[6]};
\node[anchor=center] at (7.5, 0.5) {[7]};

%fill=lightgray
\node[draw, circle] (a) at (1,2.5) {[0,1]};
\path[draw,thick,-] (a) -- (0.5,1);
\path[draw,thick,-] (a) -- (1.5,1);
\node[draw, circle,minimum size=22pt] (b) at (3,2.5) {[2,3]};
\path[draw,thick,-] (b) -- (2.5,1);
\path[draw,thick,-] (b) -- (3.5,1);
\node[draw, circle,minimum size=22pt] (c) at (5,2.5) {[4,5]};
\path[draw,thick,-] (c) -- (4.5,1);
\path[draw,thick,-] (c) -- (5.5,1);
\node[draw, circle,minimum size=22pt] (d) at (7,2.5) {[6,7]};
\path[draw,thick,-] (d) -- (6.5,1);
\path[draw,thick,-] (d) -- (7.5,1);

\node[draw, circle] (i) at (2,4.5) {[0,3]};
\path[draw,thick,-] (i) -- (a);
\path[draw,thick,-] (i) -- (b);
\node[draw, circle] (j) at (6,4.5) {[4,7]};
\path[draw,thick,-] (j) -- (c);
\path[draw,thick,-] (j) -- (d);

\node[draw, circle] (m) at (4,6.5) {[0,7]};
\path[draw,thick,-] (m) -- (i);
\path[draw,thick,-] (m) -- (j);
\end{tikzpicture}
\end{center}
\end{frame}

\subsection{Ejemplos de Queries y Updates}

\begin{frame}{Guardado de Segment Tree en Memoria}
\begin{center}
\begin{tikzpicture}[scale=0.7]
\draw (0,0) grid (15,1);

\node at (0.5,0.5) {$39$};
\node at (1.5,0.5) {$22$};
\node at (2.5,0.5) {$17$};
\node at (3.5,0.5) {$13$};
\node at (4.5,0.5) {$9$};
\node at (5.5,0.5) {$9$};
\node at (6.5,0.5) {$8$};
\node at (7.5,0.5) {$5$};
\node at (8.5,0.5) {$8$};
\node at (9.5,0.5) {$6$};
\node at (10.5,0.5) {$3$};
\node at (11.5,0.5) {$2$};
\node at (12.5,0.5) {$7$};
\node at (13.5,0.5) {$2$};
\node at (14.5,0.5) {$6$};

\footnotesize
\node at (0.5,1.4) {$1$};
\node at (1.5,1.4) {$2$};
\node at (2.5,1.4) {$3$};
\node at (3.5,1.4) {$4$};
\node at (4.5,1.4) {$5$};
\node at (5.5,1.4) {$6$};
\node at (6.5,1.4) {$7$};
\node at (7.5,1.4) {$8$};
\node at (8.5,1.4) {$9$};
\node at (9.5,1.4) {$10$};
\node at (10.5,1.4) {$11$};
\node at (11.5,1.4) {$12$};
\node at (12.5,1.4) {$13$};
\node at (13.5,1.4) {$14$};
\node at (14.5,1.4) {$15$};
\end{tikzpicture}
\end{center}
\end{frame}

\begin{frame}{Ejemplo de Query}
\begin{center}
\begin{tikzpicture}[scale=0.7]
\fill[color=yellow] (3,0) rectangle (8,1);
\draw (0,0) grid (8,1);

\node[anchor=center] at (0.5, 0.5) {5};
\node[anchor=center] at (1.5, 0.5) {8};
\node[anchor=center] at (2.5, 0.5) {6};
\node[anchor=center] at (3.5, 0.5) {3};
\node[anchor=center] at (4.5, 0.5) {2};
\node[anchor=center] at (5.5, 0.5) {7};
\node[anchor=center] at (6.5, 0.5) {2};
\node[anchor=center] at (7.5, 0.5) {6};

\node[draw, circle] (a) at (1,2.5) {13};
\path[draw,thick,-] (a) -- (0.5,1);
\path[draw,thick,-] (a) -- (1.5,1);
\node[draw, circle,minimum size=22pt] (b) at (3,2.5) {9};
\path[draw,thick,-] (b) -- (2.5,1);
\path[draw,thick,-] (b) -- (3.5,1);
\node[draw, circle,minimum size=22pt] (c) at (5,2.5) {9};
\path[draw,thick,-] (c) -- (4.5,1);
\path[draw,thick,-] (c) -- (5.5,1);
\node[draw, circle,minimum size=22pt] (d) at (7,2.5) {8};
\path[draw,thick,-] (d) -- (6.5,1);
\path[draw,thick,-] (d) -- (7.5,1);

\node[draw, circle] (i) at (2,4.5) {22};
\path[draw,thick,-] (i) -- (a);
\path[draw,thick,-] (i) -- (b);
\node[draw, circle] (j) at (6,4.5) {17};
\path[draw,thick,-] (j) -- (c);
\path[draw,thick,-] (j) -- (d);

\node[draw, circle] (m) at (4,6.5) {39};
\path[draw,thick,-] (m) -- (i);
\path[draw,thick,-] (m) -- (j);
\end{tikzpicture}
\end{center}
Suma total $= 0$
\end{frame}

\begin{frame}{Ejemplo de Query}
\begin{center}
\begin{tikzpicture}[scale=0.7]
\fill[color=yellow] (3,0) rectangle (8,1);
\draw (0,0) grid (8,1);

\node[anchor=center] at (0.5, 0.5) {5};
\node[anchor=center] at (1.5, 0.5) {8};
\node[anchor=center] at (2.5, 0.5) {6};
\node[anchor=center] at (3.5, 0.5) {3};
\node[anchor=center] at (4.5, 0.5) {2};
\node[anchor=center] at (5.5, 0.5) {7};
\node[anchor=center] at (6.5, 0.5) {2};
\node[anchor=center] at (7.5, 0.5) {6};

\node[draw, circle] (a) at (1,2.5) {13};
\path[draw,thick,-] (a) -- (0.5,1);
\path[draw,thick,-] (a) -- (1.5,1);
\node[draw, circle,minimum size=22pt] (b) at (3,2.5) {9};
\path[draw,thick,-] (b) -- (2.5,1);
\path[draw,thick,-] (b) -- (3.5,1);
\node[draw, circle,minimum size=22pt] (c) at (5,2.5) {9};
\path[draw,thick,-] (c) -- (4.5,1);
\path[draw,thick,-] (c) -- (5.5,1);
\node[draw, circle,minimum size=22pt] (d) at (7,2.5) {8};
\path[draw,thick,-] (d) -- (6.5,1);
\path[draw,thick,-] (d) -- (7.5,1);

\node[draw, circle] (i) at (2,4.5) {22};
\path[draw,thick,-] (i) -- (a);
\path[draw,thick,-] (i) -- (b);
\node[draw, circle] (j) at (6,4.5) {17};
\path[draw,thick,-] (j) -- (c);
\path[draw,thick,-] (j) -- (d);

\node[draw, circle,fill=lightgray] (m) at (4,6.5) {39};
\path[draw,thick,-] (m) -- (i);
\path[draw,thick,-] (m) -- (j);
\end{tikzpicture}
\end{center}
Suma total $= 0$
\end{frame}

\begin{frame}{Ejemplo de Query}
\begin{center}
\begin{tikzpicture}[scale=0.7]
\fill[color=yellow] (3,0) rectangle (8,1);
\draw (0,0) grid (8,1);

\node[anchor=center] at (0.5, 0.5) {5};
\node[anchor=center] at (1.5, 0.5) {8};
\node[anchor=center] at (2.5, 0.5) {6};
\node[anchor=center] at (3.5, 0.5) {3};
\node[anchor=center] at (4.5, 0.5) {2};
\node[anchor=center] at (5.5, 0.5) {7};
\node[anchor=center] at (6.5, 0.5) {2};
\node[anchor=center] at (7.5, 0.5) {6};

\node[draw, circle] (a) at (1,2.5) {13};
\path[draw,thick,-] (a) -- (0.5,1);
\path[draw,thick,-] (a) -- (1.5,1);
\node[draw, circle,minimum size=22pt] (b) at (3,2.5) {9};
\path[draw,thick,-] (b) -- (2.5,1);
\path[draw,thick,-] (b) -- (3.5,1);
\node[draw, circle,minimum size=22pt] (c) at (5,2.5) {9};
\path[draw,thick,-] (c) -- (4.5,1);
\path[draw,thick,-] (c) -- (5.5,1);
\node[draw, circle,minimum size=22pt] (d) at (7,2.5) {8};
\path[draw,thick,-] (d) -- (6.5,1);
\path[draw,thick,-] (d) -- (7.5,1);

\node[draw, circle, fill=lightgray] (i) at (2,4.5) {22};
\path[draw,thick,-] (i) -- (a);
\path[draw,thick,-] (i) -- (b);
\node[draw, circle, fill=lightgray] (j) at (6,4.5) {17};
\path[draw,thick,-] (j) -- (c);
\path[draw,thick,-] (j) -- (d);

\node[draw, circle] (m) at (4,6.5) {39};
\path[draw,thick,-] (m) -- (i);
\path[draw,thick,-] (m) -- (j);
\end{tikzpicture}
\end{center}
Suma total $= 0$
\end{frame}

\begin{frame}{Ejemplo de Query}
\begin{center}
\begin{tikzpicture}[scale=0.7]
\fill[color=yellow] (3,0) rectangle (8,1);
\draw (0,0) grid (8,1);

\node[anchor=center] at (0.5, 0.5) {5};
\node[anchor=center] at (1.5, 0.5) {8};
\node[anchor=center] at (2.5, 0.5) {6};
\node[anchor=center] at (3.5, 0.5) {3};
\node[anchor=center] at (4.5, 0.5) {2};
\node[anchor=center] at (5.5, 0.5) {7};
\node[anchor=center] at (6.5, 0.5) {2};
\node[anchor=center] at (7.5, 0.5) {6};

\node[draw, circle, fill=lightgray] (a) at (1,2.5) {13};
\path[draw,thick,-] (a) -- (0.5,1);
\path[draw,thick,-] (a) -- (1.5,1);
\node[draw, circle, minimum size=22pt, fill=lightgray] (b) at (3,2.5) {9};
\path[draw,thick,-] (b) -- (2.5,1);
\path[draw,thick,-] (b) -- (3.5,1);
\node[draw, circle,minimum size=22pt] (c) at (5,2.5) {9};
\path[draw,thick,-] (c) -- (4.5,1);
\path[draw,thick,-] (c) -- (5.5,1);
\node[draw, circle,minimum size=22pt] (d) at (7,2.5) {8};
\path[draw,thick,-] (d) -- (6.5,1);
\path[draw,thick,-] (d) -- (7.5,1);

\node[draw, circle] (i) at (2,4.5) {22};
\path[draw,thick,-] (i) -- (a);
\path[draw,thick,-] (i) -- (b);
\node[draw, circle, fill=green] (j) at (6,4.5) {17};
\path[draw,thick,-] (j) -- (c);
\path[draw,thick,-] (j) -- (d);

\node[draw, circle] (m) at (4,6.5) {39};
\path[draw,thick,-] (m) -- (i);
\path[draw,thick,-] (m) -- (j);
\end{tikzpicture}
\end{center}
Suma total $= 17$
\end{frame}

\begin{frame}{Ejemplo de Query}
\begin{center}
\begin{tikzpicture}[scale=0.7]
\fill[color=green] (3,0) rectangle (4,1);
\draw (0,0) grid (8,1);
\fill[color=yellow] (4,0) rectangle (8,1);
\draw (0,0) grid (8,1);

\node[anchor=center] at (0.5, 0.5) {5};
\node[anchor=center] at (1.5, 0.5) {8};
\node[anchor=center] at (2.5, 0.5) {6};
\node[anchor=center] at (3.5, 0.5) {3};
\node[anchor=center] at (4.5, 0.5) {2};
\node[anchor=center] at (5.5, 0.5) {7};
\node[anchor=center] at (6.5, 0.5) {2};
\node[anchor=center] at (7.5, 0.5) {6};

\node[draw, circle] (a) at (1,2.5) {13};
\path[draw,thick,-] (a) -- (0.5,1);
\path[draw,thick,-] (a) -- (1.5,1);
\node[draw, circle, minimum size=22pt] (b) at (3,2.5) {9};
\path[draw,thick,-] (b) -- (2.5,1);
\path[draw,thick,-] (b) -- (3.5,1);
\node[draw, circle,minimum size=22pt] (c) at (5,2.5) {9};
\path[draw,thick,-] (c) -- (4.5,1);
\path[draw,thick,-] (c) -- (5.5,1);
\node[draw, circle,minimum size=22pt] (d) at (7,2.5) {8};
\path[draw,thick,-] (d) -- (6.5,1);
\path[draw,thick,-] (d) -- (7.5,1);

\node[draw, circle] (i) at (2,4.5) {22};
\path[draw,thick,-] (i) -- (a);
\path[draw,thick,-] (i) -- (b);
\node[draw, circle, fill=green] (j) at (6,4.5) {17};
\path[draw,thick,-] (j) -- (c);
\path[draw,thick,-] (j) -- (d);

\node[draw, circle] (m) at (4,6.5) {39};
\path[draw,thick,-] (m) -- (i);
\path[draw,thick,-] (m) -- (j);
\end{tikzpicture}
\end{center}
Suma total $= 20$
\end{frame}

\begin{frame}{Ejemplo de Update}
\begin{center}
\begin{tikzpicture}[scale=0.7]
\fill[color=lightgray] (6,0) rectangle (7,1);
\draw (0,0) grid (8,1);

\node[anchor=center] at (0.5, 0.5) {5};
\node[anchor=center] at (1.5, 0.5) {8};
\node[anchor=center] at (2.5, 0.5) {6};
\node[anchor=center] at (3.5, 0.5) {3};
\node[anchor=center] at (4.5, 0.5) {2};
\node[anchor=center] at (5.5, 0.5) {7};
\node[anchor=center] at (6.5, 0.5) {2};
\node[anchor=center] at (7.5, 0.5) {6};

\node[draw, circle] (a) at (1,2.5) {13};
\path[draw,thick,-] (a) -- (0.5,1);
\path[draw,thick,-] (a) -- (1.5,1);
\node[draw, circle,minimum size=22pt] (b) at (3,2.5) {9};
\path[draw,thick,-] (b) -- (2.5,1);
\path[draw,thick,-] (b) -- (3.5,1);
\node[draw, circle,minimum size=22pt] (c) at (5,2.5) {9};
\path[draw,thick,-] (c) -- (4.5,1);
\path[draw,thick,-] (c) -- (5.5,1);
\node[draw, circle,minimum size=22pt] (d) at (7,2.5) {8};
\path[draw,thick,-] (d) -- (6.5,1);
\path[draw,thick,-] (d) -- (7.5,1);

\node[draw, circle] (i) at (2,4.5) {22};
\path[draw,thick,-] (i) -- (a);
\path[draw,thick,-] (i) -- (b);
\node[draw, circle] (j) at (6,4.5) {17};
\path[draw,thick,-] (j) -- (c);
\path[draw,thick,-] (j) -- (d);

\node[draw, circle] (m) at (4,6.5) {39};
\path[draw,thick,-] (m) -- (i);
\path[draw,thick,-] (m) -- (j);
\end{tikzpicture}
\end{center}
Actualizaremos la posición 6 del arreglo con el valor 10.
\end{frame}

\begin{frame}{Ejemplo de Update}
\begin{center}
\begin{tikzpicture}[scale=0.7]
\fill[color=lightgray] (6,0) rectangle (7,1);
\draw (0,0) grid (8,1);

\node[anchor=center] at (0.5, 0.5) {5};
\node[anchor=center] at (1.5, 0.5) {8};
\node[anchor=center] at (2.5, 0.5) {6};
\node[anchor=center] at (3.5, 0.5) {3};
\node[anchor=center] at (4.5, 0.5) {2};
\node[anchor=center] at (5.5, 0.5) {7};
\node[anchor=center] at (6.5, 0.5) {10};
\node[anchor=center] at (7.5, 0.5) {6};

\node[draw, circle] (a) at (1,2.5) {13};
\path[draw,thick,-] (a) -- (0.5,1);
\path[draw,thick,-] (a) -- (1.5,1);
\node[draw, circle,minimum size=22pt] (b) at (3,2.5) {9};
\path[draw,thick,-] (b) -- (2.5,1);
\path[draw,thick,-] (b) -- (3.5,1);
\node[draw, circle,minimum size=22pt] (c) at (5,2.5) {9};
\path[draw,thick,-] (c) -- (4.5,1);
\path[draw,thick,-] (c) -- (5.5,1);
\node[draw, circle,minimum size=22pt] (d) at (7,2.5) {8};
\path[draw,thick,-] (d) -- (6.5,1);
\path[draw,thick,-] (d) -- (7.5,1);

\node[draw, circle] (i) at (2,4.5) {22};
\path[draw,thick,-] (i) -- (a);
\path[draw,thick,-] (i) -- (b);
\node[draw, circle] (j) at (6,4.5) {17};
\path[draw,thick,-] (j) -- (c);
\path[draw,thick,-] (j) -- (d);

\node[draw, circle] (m) at (4,6.5) {39};
\path[draw,thick,-] (m) -- (i);
\path[draw,thick,-] (m) -- (j);
\end{tikzpicture}
\end{center}
Actualizamos nodo $14$
\end{frame}

\begin{frame}{Ejemplo de Update}
\begin{center}
\begin{tikzpicture}[scale=0.7]
\draw (0,0) grid (8,1);

\node[anchor=center] at (0.5, 0.5) {5};
\node[anchor=center] at (1.5, 0.5) {8};
\node[anchor=center] at (2.5, 0.5) {6};
\node[anchor=center] at (3.5, 0.5) {3};
\node[anchor=center] at (4.5, 0.5) {2};
\node[anchor=center] at (5.5, 0.5) {7};
\node[anchor=center] at (6.5, 0.5) {10};
\node[anchor=center] at (7.5, 0.5) {6};

\node[draw, circle] (a) at (1,2.5) {13};
\path[draw,thick,-] (a) -- (0.5,1);
\path[draw,thick,-] (a) -- (1.5,1);
\node[draw, circle,minimum size=22pt] (b) at (3,2.5) {9};
\path[draw,thick,-] (b) -- (2.5,1);
\path[draw,thick,-] (b) -- (3.5,1);
\node[draw, circle,minimum size=22pt] (c) at (5,2.5) {9};
\path[draw,thick,-] (c) -- (4.5,1);
\path[draw,thick,-] (c) -- (5.5,1);
\node[draw, circle,minimum size=22pt, fill=lightgray] (d) at (7,2.5) {16};
\path[draw,thick,-] (d) -- (6.5,1);
\path[draw,thick,-] (d) -- (7.5,1);

\node[draw, circle] (i) at (2,4.5) {22};
\path[draw,thick,-] (i) -- (a);
\path[draw,thick,-] (i) -- (b);
\node[draw, circle] (j) at (6,4.5) {17};
\path[draw,thick,-] (j) -- (c);
\path[draw,thick,-] (j) -- (d);

\node[draw, circle] (m) at (4,6.5) {39};
\path[draw,thick,-] (m) -- (i);
\path[draw,thick,-] (m) -- (j);
\end{tikzpicture}
\end{center}
Actualizamos nodo $7$ $\lfloor14/2\rfloor$
\end{frame}

\begin{frame}{Ejemplo de Update}
\begin{center}
\begin{tikzpicture}[scale=0.7]
\draw (0,0) grid (8,1);

\node[anchor=center] at (0.5, 0.5) {5};
\node[anchor=center] at (1.5, 0.5) {8};
\node[anchor=center] at (2.5, 0.5) {6};
\node[anchor=center] at (3.5, 0.5) {3};
\node[anchor=center] at (4.5, 0.5) {2};
\node[anchor=center] at (5.5, 0.5) {7};
\node[anchor=center] at (6.5, 0.5) {10};
\node[anchor=center] at (7.5, 0.5) {6};

\node[draw, circle] (a) at (1,2.5) {13};
\path[draw,thick,-] (a) -- (0.5,1);
\path[draw,thick,-] (a) -- (1.5,1);
\node[draw, circle,minimum size=22pt] (b) at (3,2.5) {9};
\path[draw,thick,-] (b) -- (2.5,1);
\path[draw,thick,-] (b) -- (3.5,1);
\node[draw, circle,minimum size=22pt] (c) at (5,2.5) {9};
\path[draw,thick,-] (c) -- (4.5,1);
\path[draw,thick,-] (c) -- (5.5,1);
\node[draw, circle,minimum size=22pt] (d) at (7,2.5) {16};
\path[draw,thick,-] (d) -- (6.5,1);
\path[draw,thick,-] (d) -- (7.5,1);

\node[draw, circle] (i) at (2,4.5) {22};
\path[draw,thick,-] (i) -- (a);
\path[draw,thick,-] (i) -- (b);
\node[draw, circle, fill=lightgray] (j) at (6,4.5) {25};
\path[draw,thick,-] (j) -- (c);
\path[draw,thick,-] (j) -- (d);

\node[draw, circle] (m) at (4,6.5) {39};
\path[draw,thick,-] (m) -- (i);
\path[draw,thick,-] (m) -- (j);
\end{tikzpicture}
\end{center}
Actualizamos nodo $3$ $\lfloor7/2\rfloor$
\end{frame}

\begin{frame}{Ejemplo de Update}
\begin{center}
\begin{tikzpicture}[scale=0.7]
\draw (0,0) grid (8,1);

\node[anchor=center] at (0.5, 0.5) {5};
\node[anchor=center] at (1.5, 0.5) {8};
\node[anchor=center] at (2.5, 0.5) {6};
\node[anchor=center] at (3.5, 0.5) {3};
\node[anchor=center] at (4.5, 0.5) {2};
\node[anchor=center] at (5.5, 0.5) {7};
\node[anchor=center] at (6.5, 0.5) {10};
\node[anchor=center] at (7.5, 0.5) {6};

\node[draw, circle] (a) at (1,2.5) {13};
\path[draw,thick,-] (a) -- (0.5,1);
\path[draw,thick,-] (a) -- (1.5,1);
\node[draw, circle,minimum size=22pt] (b) at (3,2.5) {9};
\path[draw,thick,-] (b) -- (2.5,1);
\path[draw,thick,-] (b) -- (3.5,1);
\node[draw, circle,minimum size=22pt] (c) at (5,2.5) {9};
\path[draw,thick,-] (c) -- (4.5,1);
\path[draw,thick,-] (c) -- (5.5,1);
\node[draw, circle,minimum size=22pt] (d) at (7,2.5) {16};
\path[draw,thick,-] (d) -- (6.5,1);
\path[draw,thick,-] (d) -- (7.5,1);

\node[draw, circle] (i) at (2,4.5) {22};
\path[draw,thick,-] (i) -- (a);
\path[draw,thick,-] (i) -- (b);
\node[draw, circle] (j) at (6,4.5) {25};
\path[draw,thick,-] (j) -- (c);
\path[draw,thick,-] (j) -- (d);

\node[draw, circle, fill=lightgray] (m) at (4,6.5) {47};
\path[draw,thick,-] (m) -- (i);
\path[draw,thick,-] (m) -- (j);
\end{tikzpicture}
\end{center}
Actualizamos nodo $1$ $\lfloor3/2\rfloor$
\end{frame}

\subsection{Código}

\begin{frame}[fragile]{Version Iterativa}
\begin{lstlisting}
int retrieve(int a, int b) {
    a += n; b += n;
    int s = 0;
    while (a <= b) {
        if (a%2 == 1) s += tree[a++];
        if (b%2 == 0) s += tree[b--];
        a /= 2; b /= 2;
    }
    return s;
}

void update(int k, int x) {
    k += n, tree[k] += x;
    for (k /= 2; k >= 1; k /= 2) {
        tree[k] = tree[2*k]+tree[2*k+1];
    }
}
\end{lstlisting}
\end{frame}

\begin{frame}[fragile]{Version Recursiva}
\begin{lstlisting}
int sum(int a, int b, int k, int x, int y) {
    if (b < x || a > y) return 0;
    if (a <= x && y <= b) return tree[k];
    int d = (x+y)/2;
    return sum(a,b,2*k,x,d) + sum(a,b,2*k+1,d+1,y);
}
      
void update(int pos, int k, int x, int y, int val) {
    if (x == y) { segTree[k] = val; return; }
    int d = (x + y) / 2;
    if (pos <= d) update(pos, k*2, x, d, val);
    else update(pos, k*2+1, d+1, y, val);
    segTree[k] = segTree[k*2] + segTree[k*2+1];
}

\end{lstlisting}
\end{frame}

\subsection{Complejidad}

\begin{frame}{Complejidad}
\begin{itemize}
    \item Construcción: $O(n)$ (De todas formas, solemos construirlo usando la operación update)
    \item Query: $O(\log n)$
    \item Update: $O(\log n)$
    \item Memoria: $O(2n)$
\end{itemize}
Puede notarse que el árbol que construimos es un árbol binario completo, por lo que la cantidad de nodos es $2n-1$, y tiene una profundidad de $log_2(n)$.
\end{frame}

\begin{frame}{Entendiendo la Complejidad de Query}
Es fácil creer que la complejidad de query es mayor a $O(\log n)$, ya que en cada nivel del árbol se podría llamar a los dos nodos hijos. Pero, en total, notemos que en cada nivel solo se van a estar llamando como máximo $4$ nodos en su versión recursiva, y $2$ en la versión iterativa. Veamos el caso recursivo, que es el más confuso.

\begin{itemize}
    \item El nodo está afuera del rango: no se llama a ningún hijo.
    \item El nodo está completamente dentro del rango: no se llama a ningún hijo.
    \item El nodo está parcialmente dentro del rango: se llama a los dos hijos. Esto puede ocurrir hasta dos veces por paso, ya que el intervalo es contiguo.
\end{itemize}
\end{frame}

\subsection{Resolución de un Problema}

\begin{frame}{Resolución del Problema}
Veamos el Código de ``Dynamic Range Sum Queries''.

\url{https://cses.fi/problemset/task/1648}
\end{frame}

\section{Generalización del Segment Tree}

\begin{frame}{Pregunta Motivadora}
La estructura de datos que trabajamos anteriormente, ¿sólo nos permite trabajar con la suma?

\pause
¡No! Podemos generalizarla para cualquier operación que sea asociativa y tenga un elemento neutro. A este tipo de operaciones las llamamos \textbf{monoides}.
\end{frame}

\subsection{Monoides}

\begin{frame}{Definición de Monoide}
Un Monoide es una estructura algebraica con una operación binaria, que es asociativa y tiene elemento neutro. Un Monoide se lo puede denotar como $(A, \oplus)$ y en lenguaje formal se define como:
\begin{itemize}
    \item $\forall a,b,c \in A, (a \oplus b) \oplus c = a \oplus (b \oplus c)$ (es asociativa)
    \item $\exists !e \in A, \forall a \in A, a \oplus e = e \oplus a = a$ (existe neutro)
    \item $\forall a,b \in A, a \oplus b \in A$ (operación cerrada)
\end{itemize}
\end{frame}

\begin{frame}{Ejemplos de Monoide}
Estos son algunos ejemolos de Monoides, que pueden ser interesantes para resolver problemas de Programación Competitiva:
\begin{itemize}
    \item $(\mathbb{N}, +)$
    \item $(\mathbb{N}, \times)$
    \item $(\mathbb{N}, min)$
    \item $(\mathbb{N}, max)$
    \item $(\mathbb{N}, gcd)$
    \item $(\mathbb{N}, lcm)$
    \item $(\mathbb{N}, \land)$
    \item $(\mathbb{N}, \lor)$
    \item $(\mathbb{N}, \oplus)$, donde $\oplus$ es el XOR
    \item $(\text{string}, +)$, donde $+$ es la concatenación
\end{itemize}
\end{frame}

\subsection{Implementación Generalizada del Segment Tree}

\begin{frame}{Implementación Generalizada}
Ahora, vamos a ver en C++ una implementación generalizada del Segment Tree, que nos permita trabajar con cualquier Monoide. Esta idea es gran cortesía de Tarche, que hizo que esté en nuestro Notebook para ICPC.
\end{frame}

\subsection{Resolviendo Más Problemas}
\begin{frame}{Resolviendo Más Problemas}
Resolvamos los siguientes problemas de CSES con la implementación de Monoide:
\begin{itemize}
    \item \url{https://cses.fi/problemset/task/1649}
    \item \url{https://cses.fi/problemset/task/1650}
\end{itemize}
\end{frame}

\section{Más Problemas de Segment Tree}

\subsection{Hotel Queries}

\begin{frame}{Hotel Queries}
\begin{block}{CSES 1143 - Hotel Queries}
Hay $n$ hoteles en una calle. Para cada hotel conoces el número de habitaciones libres. Tu tarea es asignar habitaciones de hotel para grupos de turistas. Todos los miembros de un grupo quieren alojarse en el mismo hotel.

Los grupos llegarán a usted uno tras otro y usted sabrá para cada grupo la cantidad de habitaciones que necesita. Siempre asignas un grupo al primer hotel que tenga suficientes habitaciones. Después de esto, el número de habitaciones libres en el hotel disminuye.

Para cada grupo de turistas, imprima el número del hotel al que fueron asignados.
\end{block}

\url{https://cses.fi/problemset/task/1143}
\end{frame}

\begin{frame}{Recorriendo Alternativamente el Segment Tree}
Para resolver el problema, vamos a usar el Segment Tree de máximos como una estructura para hacer Binary Search. La idea es encontrar el primer hotel que tenga suficientes habitaciones para el grupo de turistas en $O(\log n)$.

\pause
Partimos del primer nodo (la raíz), y recorremos el de la izquierda si el nodo contiene un número mayor o igual a la cantidad de habitaciones que necesitamos, o el de la derecha si el nodo contiene un número menor.

\pause
Luego, actualizamos el Segment Tree para que tenga menos habitaciones libres. ¡Veamos el código!
\end{frame}

\subsection{Salary Queries}
\begin{frame}{Salary Queries}
\begin{block}{CSES 1144 - Salary Queries}
Una compañía tiene $n$ empleados con ciertos salarios. Tu tarea es procesar salarios, pudiendo preguntar cuales eran los salarios de los empleados en un rango, y actualizar el salario de un empleado.
\end{block}

\url{https://cses.fi/problemset/task/1144}
\end{frame}

\begin{frame}{Solución 1 - Segment Tree Dinámico}
El gran problema que tiene este ejercicio es que los salarios pueden ser hasta $1e9$, por lo que si queremos guardar un Segment Tree en el que cada posición guarde cuántas personas tienen ese salario (para hacer una query de rangos) requeriríamos muchísima memoria.

\pause
Para resolver esto, podemos usar un Segment Tree Dinámico, que solo guarde los nodos que necesitemos. De esta forma, la complejidad de memoria es $O(n\log n)$ en la cantidad de updates, y la complejidad de tiempo es $O(n\log n)$ para construirlo, $O(\log n)$ para cada query y $O(\log n)$ para cada update.
\end{frame}

\begin{frame}{Solución 2 - Compresión de Coordenadas}
Otra forma de resolverlo (que es una idea muy común en programación competitiva cuando los problemas son offline) es utilizar compresión de coordenadas.

\pause
Lo que se hace es guardar los salarios actuales y posibles a futuro (leyendo todas las queries) en un arreglo, y luego comprimirlos para que sean números del $0$ al $n$, asignando a cada valor posible un número comprimido. De esta forma, vamos a tener como mucho $2n$ valores posibles, y podemos guardar en un Segment Tree tradicional.
\end{frame}

\subsection{Otros Problemas Sugeridos}

\begin{frame}{Otros Problemas}
\begin{enumerate}
    \item Range Update Queries: \url{https://cses.fi/problemset/task/1651} (Updates en rango y queries en punto)
    \item List Removals: \url{https://cses.fi/problemset/task/1749} (Búsqueda binaria en el segment tree)
    \item Distinct Value Queries: \url{https://cses.fi/problemset/task/1734/} (Segment Tree dinámico o compresión de coordenadas)
    \item Prefix Sum Queries: \url{https://cses.fi/problemset/task/2166/} (Operaciones con nodos más complejas) 
\end{enumerate}
\end{frame}

\section{Lazy Segment Tree}
\subsection{Introducción}

\begin{frame}{Llevándolo un poco más lejos...}
Ahora queremos realizar no solo queries en rangos, sino que también updates en rangos, de forma eficiente (sin hacer un nodo a la vez). ¿Cómo podemos hacerlo?
\end{frame}

\begin{frame}{Lazy Segment Tree}
La idea es usar un Segment Tree que tenga un arreglo de lazy updates, que nos permita guardar los updates que no se han propagado a los hijos. De esta forma, podemos hacer updates en rangos en $O(\log n)$.

\pause
Para ello, vamos a hacer que la propagación sea ``perezosa'', en el sentido de que una vez que llegue se llegue a un nodo que este totalmente contenido al intervalo a actualizar, no se va a propagar a sus hijos.

\pause
Veamos un ejemplo de problema para poder entender esto mejor...
\end{frame}

\subsection{Problema de Suma}
\begin{frame}{Range Updates and Sums}
\begin{block}{CSES 1735 - Range Updates and Sums}
Tu tarea es mantener un vector de $n$ valores y procesar eficientemente los siguientes tipos de consultas:
\begin{enumerate}
    \item Aumente cada valor en el rango $[a,b]$ en $x$.
    \item Establezca cada valor en el rango $[a,b]$ en $x$.
    \item Calcule la suma de los valores en el rango $[a,b]$.
\end{enumerate}
\end{block}

\url{https://cses.fi/problemset/task/1735}
\end{frame}

\begin{frame}{Ejemplo de Lazy Segment Tree}
\begin{center}
\begin{tikzpicture}[scale=0.7]
\draw (0,0) grid (16,1);

\node[anchor=center] at (0.5, 0.5) {5};
\node[anchor=center] at (1.5, 0.5) {8};
\node[anchor=center] at (2.5, 0.5) {6};
\node[anchor=center] at (3.5, 0.5) {3};
\node[anchor=center] at (4.5, 0.5) {2};
\node[anchor=center] at (5.5, 0.5) {7};
\node[anchor=center] at (6.5, 0.5) {2};
\node[anchor=center] at (7.5, 0.5) {6};
\node[anchor=center] at (8.5, 0.5) {7};
\node[anchor=center] at (9.5, 0.5) {1};
\node[anchor=center] at (10.5, 0.5) {7};
\node[anchor=center] at (11.5, 0.5) {5};
\node[anchor=center] at (12.5, 0.5) {6};
\node[anchor=center] at (13.5, 0.5) {2};
\node[anchor=center] at (14.5, 0.5) {3};
\node[anchor=center] at (15.5, 0.5) {2};

\node[draw, circle] (a) at (1,2.5) {13/0};
\path[draw,thick,-] (a) -- (0.5,1);
\path[draw,thick,-] (a) -- (1.5,1);
\node[draw, circle,minimum size=32pt] (b) at (3,2.5) {9/0};
\path[draw,thick,-] (b) -- (2.5,1);
\path[draw,thick,-] (b) -- (3.5,1);
\node[draw, circle,minimum size=32pt] (c) at (5,2.5) {9/0};
\path[draw,thick,-] (c) -- (4.5,1);
\path[draw,thick,-] (c) -- (5.5,1);
\node[draw, circle,minimum size=32pt] (d) at (7,2.5) {8/0};
\path[draw,thick,-] (d) -- (6.5,1);
\path[draw,thick,-] (d) -- (7.5,1);
\node[draw, circle,minimum size=32pt] (e) at (9,2.5) {8/0};
\path[draw,thick,-] (e) -- (8.5,1);
\path[draw,thick,-] (e) -- (9.5,1);
\node[draw, circle] (f) at (11,2.5) {12/0};
\path[draw,thick,-] (f) -- (10.5,1);
\path[draw,thick,-] (f) -- (11.5,1);
\node[draw, circle,minimum size=32pt] (g) at (13,2.5) {8/0};
\path[draw,thick,-] (g) -- (12.5,1);
\path[draw,thick,-] (g) -- (13.5,1);
\node[draw, circle,minimum size=32pt] (h) at (15,2.5) {5/0};
\path[draw,thick,-] (h) -- (14.5,1);
\path[draw,thick,-] (h) -- (15.5,1);

\node[draw, circle] (i) at (2,4.5) {22/0};
\path[draw,thick,-] (i) -- (a);
\path[draw,thick,-] (i) -- (b);
\node[draw, circle] (j) at (6,4.5) {17/0};
\path[draw,thick,-] (j) -- (c);
\path[draw,thick,-] (j) -- (d);
\node[draw, circle] (k) at (10,4.5) {20/0};
\path[draw,thick,-] (k) -- (e);
\path[draw,thick,-] (k) -- (f);
\node[draw, circle] (l) at (14,4.5) {13/0};
\path[draw,thick,-] (l) -- (g);
\path[draw,thick,-] (l) -- (h);

\node[draw, circle] (m) at (4,6.5) {39/0};
\path[draw,thick,-] (m) -- (i);
\path[draw,thick,-] (m) -- (j);
\node[draw, circle] (n) at (12,6.5) {33/0};
\path[draw,thick,-] (n) -- (k);
\path[draw,thick,-] (n) -- (l);

\node[draw, circle] (o) at (8,8.5) {72/0};
\path[draw,thick,-] (o) -- (m);
\path[draw,thick,-] (o) -- (n);
\end{tikzpicture}
\end{center}

\end{frame}

\begin{frame}{Update en Lazy Segment Tree}
\begin{center}
\begin{tikzpicture}[scale=0.7]
\fill[color=gray!50] (5,0) rectangle (6,1);
\draw (0,0) grid (16,1);

\node[anchor=center] at (0.5, 0.5) {5};
\node[anchor=center] at (1.5, 0.5) {8};
\node[anchor=center] at (2.5, 0.5) {6};
\node[anchor=center] at (3.5, 0.5) {3};
\node[anchor=center] at (4.5, 0.5) {2};
\node[anchor=center] at (5.5, 0.5) {9};
\node[anchor=center] at (6.5, 0.5) {2};
\node[anchor=center] at (7.5, 0.5) {6};
\node[anchor=center] at (8.5, 0.5) {7};
\node[anchor=center] at (9.5, 0.5) {1};
\node[anchor=center] at (10.5, 0.5) {7};
\node[anchor=center] at (11.5, 0.5) {5};
\node[anchor=center] at (12.5, 0.5) {6};
\node[anchor=center] at (13.5, 0.5) {2};
\node[anchor=center] at (14.5, 0.5) {3};
\node[anchor=center] at (15.5, 0.5) {2};

\node[draw, circle] (a) at (1,2.5) {13/0};
\path[draw,thick,-] (a) -- (0.5,1);
\path[draw,thick,-] (a) -- (1.5,1);
\node[draw, circle,minimum size=32pt] (b) at (3,2.5) {9/0};
\path[draw,thick,-] (b) -- (2.5,1);
\path[draw,thick,-] (b) -- (3.5,1);
\node[draw, circle,minimum size=32pt] (c) at (5,2.5) {11/0};
\path[draw,thick,-] (c) -- (4.5,1);
\path[draw,thick,-] (c) -- (5.5,1);
\node[draw, circle,fill=gray!50,minimum size=32pt] (d) at (7,2.5) {8/2};
\path[draw,thick,-] (d) -- (6.5,1);
\path[draw,thick,-] (d) -- (7.5,1);
\node[draw, circle,minimum size=32pt] (e) at (9,2.5) {8/0};
\path[draw,thick,-] (e) -- (8.5,1);
\path[draw,thick,-] (e) -- (9.5,1);
\node[draw, circle] (f) at (11,2.5) {12/0};
\path[draw,thick,-] (f) -- (10.5,1);
\path[draw,thick,-] (f) -- (11.5,1);
\node[draw, circle,fill=gray!50,minimum size=32pt] (g) at (13,2.5) {8/2};
\path[draw,thick,-] (g) -- (12.5,1);
\path[draw,thick,-] (g) -- (13.5,1);
\node[draw, circle,minimum size=32pt] (h) at (15,2.5) {5/0};
\path[draw,thick,-] (h) -- (14.5,1);
\path[draw,thick,-] (h) -- (15.5,1);

\node[draw, circle] (i) at (2,4.5) {22/0};
\path[draw,thick,-] (i) -- (a);
\path[draw,thick,-] (i) -- (b);
\node[draw, circle] (j) at (6,4.5) {23/0};
\path[draw,thick,-] (j) -- (c);
\path[draw,thick,-] (j) -- (d);
\node[draw, circle,fill=gray!50] (k) at (10,4.5) {20/2};
\path[draw,thick,-] (k) -- (e);
\path[draw,thick,-] (k) -- (f);
\node[draw, circle] (l) at (14,4.5) {17/0};
\path[draw,thick,-] (l) -- (g);
\path[draw,thick,-] (l) -- (h);

\node[draw, circle] (m) at (4,6.5) {45/0};
\path[draw,thick,-] (m) -- (i);
\path[draw,thick,-] (m) -- (j);
\node[draw, circle] (n) at (12,6.5) {45/0};
\path[draw,thick,-] (n) -- (k);
\path[draw,thick,-] (n) -- (l);

\node[draw, circle] (o) at (8,8.5) {90/0};
\path[draw,thick,-] (o) -- (m);
\path[draw,thick,-] (o) -- (n);

\path[draw=red,thick,->,line width=2pt] (o) -- (m);
\path[draw=red,thick,->,line width=2pt] (o) -- (n);

\path[draw=red,thick,->,line width=2pt] (m) -- (j);
\path[draw=red,thick,->,line width=2pt] (j) -- (c);
\path[draw=red,thick,->,line width=2pt] (j) -- (d);
\path[draw=red,thick,->,line width=2pt] (c) -- (5.5,1);

\path[draw=red,thick,->,line width=2pt] (n) -- (k);
\path[draw=red,thick,->,line width=2pt] (n) -- (l);

\path[draw=red,thick,->,line width=2pt] (l) -- (g);

\node at (5.5,-0.75) {$a$};
\node at (13.5,-0.75) {$b$};
\end{tikzpicture}
\end{center}

\end{frame}

\begin{frame}{Propagando al hacer Query}
\begin{center}
\begin{tikzpicture}[scale=0.7]
\draw (0,0) grid (16,1);

\node[anchor=center] at (0.5, 0.5) {5};
\node[anchor=center] at (1.5, 0.5) {8};
\node[anchor=center] at (2.5, 0.5) {6};
\node[anchor=center] at (3.5, 0.5) {3};
\node[anchor=center] at (4.5, 0.5) {2};
\node[anchor=center] at (5.5, 0.5) {9};
\node[anchor=center] at (6.5, 0.5) {2};
\node[anchor=center] at (7.5, 0.5) {6};
\node[anchor=center] at (8.5, 0.5) {7};
\node[anchor=center] at (9.5, 0.5) {1};
\node[anchor=center] at (10.5, 0.5) {7};
\node[anchor=center] at (11.5, 0.5) {5};
\node[anchor=center] at (12.5, 0.5) {6};
\node[anchor=center] at (13.5, 0.5) {2};
\node[anchor=center] at (14.5, 0.5) {3};
\node[anchor=center] at (15.5, 0.5) {2};

\node[draw, circle] (a) at (1,2.5) {13/0};
\path[draw,thick,-] (a) -- (0.5,1);
\path[draw,thick,-] (a) -- (1.5,1);
\node[draw, circle,minimum size=32pt] (b) at (3,2.5) {9/0};
\path[draw,thick,-] (b) -- (2.5,1);
\path[draw,thick,-] (b) -- (3.5,1);
\node[draw, circle,minimum size=32pt] (c) at (5,2.5) {11/0};
\path[draw,thick,-] (c) -- (4.5,1);
\path[draw,thick,-] (c) -- (5.5,1);
\node[draw, circle,minimum size=32pt] (d) at (7,2.5) {8/2};
\path[draw,thick,-] (d) -- (6.5,1);
\path[draw,thick,-] (d) -- (7.5,1);
\node[draw, circle,minimum size=32pt] (e) at (9,2.5) {8/2};
\path[draw,thick,-] (e) -- (8.5,1);
\path[draw,thick,-] (e) -- (9.5,1);
\node[draw, circle,fill=gray!50,] (f) at (11,2.5) {12/2};
\path[draw,thick,-] (f) -- (10.5,1);
\path[draw,thick,-] (f) -- (11.5,1);
\node[draw, circle,fill=gray!50,minimum size=32pt] (g) at (13,2.5) {8/2};
\path[draw,thick,-] (g) -- (12.5,1);
\path[draw,thick,-] (g) -- (13.5,1);
\node[draw, circle,minimum size=32pt] (h) at (15,2.5) {5/0};
\path[draw,thick,-] (h) -- (14.5,1);
\path[draw,thick,-] (h) -- (15.5,1);

\node[draw, circle] (i) at (2,4.5) {22/0};
\path[draw,thick,-] (i) -- (a);
\path[draw,thick,-] (i) -- (b);
\node[draw, circle] (j) at (6,4.5) {23/0};
\path[draw,thick,-] (j) -- (c);
\path[draw,thick,-] (j) -- (d);
\node[draw, circle] (k) at (10,4.5) {28/0};
\path[draw,thick,-] (k) -- (e);
\path[draw,thick,-] (k) -- (f);
\node[draw, circle] (l) at (14,4.5) {17/0};
\path[draw,thick,-] (l) -- (g);
\path[draw,thick,-] (l) -- (h);

\node[draw, circle] (m) at (4,6.5) {45/0};
\path[draw,thick,-] (m) -- (i);
\path[draw,thick,-] (m) -- (j);
\node[draw, circle] (n) at (12,6.5) {45/0};
\path[draw,thick,-] (n) -- (k);
\path[draw,thick,-] (n) -- (l);

\node[draw, circle] (o) at (8,8.5) {90/0};
\path[draw,thick,-] (o) -- (m);
\path[draw,thick,-] (o) -- (n);

\path[draw=red,thick,->,line width=2pt] (o) -- (n);

\path[draw=red,thick,->,line width=2pt] (n) -- (k);
\path[draw=red,thick,->,line width=2pt] (n) -- (l);

\path[draw=red,thick,->,line width=2pt] (k) -- (f);
\path[draw=red,thick,->,line width=2pt] (l) -- (g);

\draw[color=blue,thick] (8,1.5) rectangle (12,5.5);

\node at (10.5,-0.75) {$a$};
\node at (13.5,-0.75) {$b$};
\end{tikzpicture}
\end{center}

\end{frame}

\begin{frame}{¿Cómo pensar un problema de Lazy Segment Tree?}
Para que un problema pueda realizarse con Lazy Segment Tree, debemos poder contar con una operación que calcule el valor de un nodo sin haber propagado los updates a sus hijos.

En este caso, lo podemos hacer guardando tres variables ($suma$, $sumaLazy$ y $setValueLazy$), de la siguiente forma:
\end{frame}

\begin{frame}{¿Cómo pensar un problema de Lazy Segment Tree?}
\begin{enumerate}
    \item Si no hay updates por hacer, el nodo vale $suma$: la suma de sus hijos.
    \pause
    \item Si hay un update de seteo, el nodo vale $setValueLazy \times (r-l+1)$: el valor que se seteo multiplicado por la cantidad de elementos en el rango.
    \pause
    \item Si hay un update de suma, el nodo vale $sumaLazy \times (r-l+1) + suma$: lo que se le va a sumar a cada uno de sus hijos por la cantidad de elementos en el rango, más la suma de los hijos.
    \pause
    \item Si hay un update de suma y un update de seteo, el nodo vale $(setValueLazy  + sumaLazy) \times (r-l+1)$: el valor que se seteo multiplicado por la cantidad de elementos en el rango.
\end{enumerate}
\end{frame}

\begin{frame}{Los pasos de un Update}
\begin{enumerate}
    \item Retornar si el rango está afuera del update.
    \pause
    \item Si el rango está contenido en el nodo, setear los valores ``lazy'' y retornar.
    \pause
    \item Si el rango está parcialmente contenido en el nodo, propagar el update a los hijos y borrar los valores ``lazy''.
    \pause
    \item Llamar a actualizar cada uno de los nodos hijos.
    \pause
    \item Actualizar el valor del nodo.
\end{enumerate}
\end{frame}


\begin{frame}{Los pasos de una Query}
\begin{enumerate}
    \item Retornar si el rango está afuera de la query.
    \pause
    \item Si el rango está contenido en el nodo, retornar el valor del nodo realizando la operación ``lazy''.
    \pause
    \item Si el rango está parcialmente contenido en el nodo, propagar el update a los hijos y borrar los valores ``lazy''.
    \pause
    \item Llamar a query cada uno de los nodos hijos.
    \pause
    \item Actualizar los valores de este nodo.
    \pause
    \item Retornar el valor del nodo.
\end{enumerate}
\end{frame}

\begin{frame}{Yendo al código...}
Veamos el código para entender cómo es que se puede programar de forma sencilla (o no tanto).
\end{frame}

\subsection{Generalizando el Lazy Segment Tree}
\begin{frame}{Generalizando el Lazy Segment Tree}
Para que una operación se pueda generalizar en un Lazy Segment tree, debe cumplir con las siguientes propiedades...
\end{frame}

\subsection{Polinomial Queries}
\begin{frame}{Polinomial Queries}
\begin{block}{CSES 1736 - Polinomial Queries}
Su tarea es mantener un vector de $n$ valores y procesar eficientemente los siguientes tipos de consultas:
\begin{enumerate}
    \item Aumente el primer valor en el rango $[a,b]$ en 1, el segundo valor en 2, el tercer valor en 3, y así sucesivamente.
    \item Calcule la suma de los valores en el rango $[a,b]$.
\end{enumerate}
\end{block}

\url{https://cses.fi/problemset/task/1736/}
\end{frame}

\begin{frame}{Variables de los Nodos}
\begin{itemize}
    \item $suma$: la suma de los valores en el rango.
    \item $lazy$: valor que contiene cuanto se le suma al primer elemento del rango que denota el nodo.
    \item $cantidad$: cuántas veces se le aplicó una operación de update lazy a ese nodo (antes de propagarlo).
\end{itemize}  
\end{frame}
    
\begin{frame}{Acción de propagación}
Para el nodo izquierdo, propagamos sumando tanto los valores $lazy$ como la $cantidad$.

\pause
Para el nodo derecho, el valor de $lazy$ se le suma $actual.lazy+actual.cantidad * mitadRangoActual$. Esto hace que se guarde nuevamente cual es el valor del primer elemento en el nodo. Luego, a $cantidad$ simplemente se le suma la $cantidad$ del nodo actual.
\end{frame}
    
\begin{frame}{Evaluación de un Nodo}
Para poder obtener el valor de un nodo, debemos realizar la siguiente cuenta:

$(sumaPoli \times actual.cantidad) + (actual.lazy \times cantidadHijos) + actual.suma$

\pause
Siendo:

$cantHijos = y - x + 1$, con $[x,y]$ el rango del nodo.
$sumaPoli = \frac{cantHijos \times (cantHijos-1)}{2}$
\end{frame}

\begin{frame}{¡Veamos el codigo!}
Leamos el código subido al juez, donde se pueden identificar estos pasos.
\end{frame}

\section{Bibliografía y Temas Relacionados}
\begin{frame}{¡Muchas gracias!}
Bibliografía:

\begin{enumerate}
    \item \url{https://cses.fi/book/book.pdf}
    \item \url{http://www.oia.unsam.edu.ar/wp-content/uploads/2017/11/segment-tree.pdf}
    \item \url{https://es.wikipedia.org/wiki/Monoide}
    \item \url{https://usaco.guide}
    \item \url{https://github.com/pllk/cphb/blob/master/chapter28.tex}
    \item \url{https://cp-algorithms.com/data_structures/segment_tree.html}
\end{enumerate}
¡Agradecimientos a los dibujos del Guide to Competitive Programming!
\end{frame}

\begin{frame}{Temas Relacionados}
\begin{enumerate}
    \item Persistent Segment Tree
    \item Fenwick Tree
    \item Segment Tree y Fenwick Tree 2D
    \item Segment Tree Lazy Creation + Lazy Update (codigo de USACO Guide)
\end{enumerate}
Se pueden profundizar en la bibliografía.
\end{frame}

\end{document}
